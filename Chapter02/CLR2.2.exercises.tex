\documentclass[a4paper,12pt]{article}
\usepackage{ amssymb }
\usepackage{mathtools}
\usepackage{amsmath}
\usepackage{amssymb}
\usepackage{listings}
\usepackage{color}
\usepackage[utf8]{inputenc}
\usepackage{amsfonts}
 \usepackage{tikz}
 \usepgflibrary{fpu}
 \usepackage{graphicx}
\DeclarePairedDelimiter\ceil{\lceil}{\rceil}
\DeclarePairedDelimiter\floor{\lfloor}{\rfloor}

\definecolor{codegreen}{rgb}{0,0.6,0}
\definecolor{codegray}{rgb}{0.5,0.5,0.5}
\definecolor{codepurple}{rgb}{0.58,0,0.82}
\definecolor{backcolour}{rgb}{0.95,0.95,0.92}
\definecolor{highlightcolor}{rgb}{0.8, 0.9, 0.9}

\lstdefinestyle{mystyle}{
    backgroundcolor=\color{backcolour},   
    commentstyle=\color{codegreen},
    keywordstyle=\color{magenta},
    numberstyle=\tiny\color{codegray},
    stringstyle=\color{codepurple},
    basicstyle=\footnotesize,
    breakatwhitespace=false,         
    breaklines=true,                 
    captionpos=b,                    
    keepspaces=true,                 
    numbers=left,                    
    numbersep=5pt,                  
    showspaces=false,                
    showstringspaces=false,
    showtabs=false,                  
    tabsize=2
}
 
 \lstset{style=mystyle}

\newcommand*\Suppressnumber{%
  \lst@AddToHook{OnNewLine}{%
    \let\thelstnumber\relax%
     \advance\c@lstnumber-\@ne\relax%
    }%
}

\begin{document}

\noindent{
\framebox{
\begin{minipage}{\dimexpr\textwidth-2\fboxsep-2\fboxrule\relax}
\vspace{2mm}
1.  Show that if $f(n)$ and $g(n)$ are monotonically increasing functions, then so are the functions $f(n) + g(n)$ and $f(g(n))$, and if $f(n)$ and $g(n)$ are in additional nonnegative, then $f(n) \cdot g(n)$ is monotonically increasing. 
\vspace{2mm}
\end{minipage}
}
}

A function $f$ is \textit{monotonically increasing} if $n \le m$ implies $f(n) \le f(m)$. Suppose that $f$ and $g$ are monotonically increasing. Then $n \le m$ implies $f(n) \le f(m)$ and $g(n) \le g(m)$, and hence $f(n) + g(n) \le f(m) + g(m)$ and so $f(n) + g(n)$ is monotonically increasing. 

Since $g$ is monotonically increasing, if $n \le m$ then $g(n) \le g(m)$. Let $n'=g(n)$ and $m'=g(m)$, so that $n' \le m'$. Since $f$ is monotonically increasing and $n' \le m'$, we have $f(n') \le f(m')$. Hence $f(g(n)) \le f(g(m))$ and $f(g(n)$ is monotonically increasing. 

Suppose that $f$ and $g$ are nonnegative, so that $f(n) \ge 0$ and $g(n) \ge 0$. Furthermore, $f$ and $g$ are monotonically increasing so that $n \le m$ implies $f(n) \le f(m)$ and $g(n) \le g(m)$. Because $f$ and $g$ are nonnegative we can multiply the inequalities to obtain $f(n) \cdot g(n) \le f(m) \cdot g(m)$. Hence $f(n) \cdot g(n)$ is monotonically increasing. 

\vspace{2mm}
\noindent{
\framebox{
\begin{minipage}{\dimexpr\textwidth-2\fboxsep-2\fboxrule\relax}
\vspace{2mm}
2.  Use the definition of $O$-notation to show that $T(n) = n^{O(1)}$ if and only if there exists a constant $k > 0$ such that $T(n) = O(n^k)$.

\vspace{2mm}
\end{minipage}
}
}

Suppose that $T(n) = n^{O(1)}$. The set $O(1)$ is defined as $O(1) = \{ f(n) :$ there exist positive constants $c$ and $n_0$ such that $0 \le f(n) \le c \cdot 1 = c$ for all $n \ge n_0$$\}$. If other words, we have $f(n) = O(1)$ only if $f$ is bounded above by some positive constant $c$. Clearly, for any positive constant $k >0$ we have $k = O(1)$, since we can always choose another positive constant $c_1>0$ such that $0 < k \le c_1$. Since $k = O(1)$, we can write $T(n) = n^k$.

Since $T(n) = n^k$, we have that $T(n) = O(n^k)$, since we can find a positive constants $c$ and $n_0$ such that $0 \le T(n) \le c \cdot n^k$ for all $n \ge n_0$. Specifically, choose $n_0=1$ and $c=2$ to establish that $T(n) = O(n^k)$. 

This establishes that $T(n) = n^{O(1)} \Rightarrow T(n) = O(n^k)$.

Suppose that $T(n) = O(n^k)$, where $k > 0$. Then we can find positive constants $c_1$ and $n_1$ such that $0 \le T(n) \le c_1\cdot n_1^k$ for all $n \ge n_1$. To demonstrate that a function $f$ is $f = O(1)$, we must find positive constants $c_2$ and $n_2$ such that $0 \le f(n) \le c_2 \cdot 1 = c_2$ for all $n \ge n_2$. Since $T(n) = O(n^k)$ we know that $T(n) \le c_1 \cdot n^k$, and to show that $T(n) = n^{O(1)}$ we must find $c_2$ and $n_2$ such that $T(n) \le n^{c_2}$ for all $n \ge n_2$. 

Let $c_2 = k+1$, hence we can write $c_1 \cdot n^k \le n^{k+1}$. Dividing both sides by $n^k$, we have $c_1 \le n$. Hence, if we choose positive constants $c_2 = k+1$ and $n_2 \ge c_1$, we have that $T(n) = n^{O(1)}$. 

This establishes that $T(n) = O(n^k) \Rightarrow T(n) = n^{O(1)}$.

\vspace{2mm}
Hence we conclude that $T(n) = n^{O(1)} \iff T(n) = O(n^k)$.

\vspace{2mm}
\noindent{
\framebox{
\begin{minipage}{\dimexpr\textwidth-2\fboxsep-2\fboxrule\relax}
\vspace{2mm}
3.  Prove equation (2.9).
\vspace{2mm}
\end{minipage}
}
}

Equation (2.9) is given as:

\[ a^{\log_b n} = n^{\log_b a}\]

We can make use of the identity $a = b^{\log_b a}$ as follows:

\[ a^{\log_b n} = b^{\log_b a \cdot \log_b n} = b^{\log_b n \cdot \log_b a} = n^{\log_b a} \]

\vspace{2mm}
\noindent{
\framebox{
\begin{minipage}{\dimexpr\textwidth-2\fboxsep-2\fboxrule\relax}
\vspace{2mm}
4.  Prove that $\lg(n!) = \Theta(n \lg n)$ and that $n! = o(n^n)$.
\vspace{2mm}
\end{minipage}
}
}

For the first part, if $n \ge 1$ we have:

\[ \lg (n!) = \sum_{k=1}^n \lg k \le n \lg n \]
Hence by choosing $n \ge 1$ and $c \ge 1$, we have $\lg(n!) = O(n \lg n)$. To prove that $\lg(n!) = \Omega(n \lg n)$, we can perform a similar calculation, utilizing only the second half of the summation:

\[ \frac{1}{2}n\lg n -\frac{1}{2}n =\left(\frac{n}{2}\right) \cdot \lg \frac{n}{2} \le \sum_{k=n/2}^n \lg k \le \sum_{k=1}^n \lg k = \lg(n!) \]

Suppose we have $n \ge 4$, then $\frac{1}{4}n\lg n \le \frac{1}{2}n \lg n - \frac{1}{2}n$ and $\frac{1}{4}n \lg n \le \lg (n!)$. Hence by choosing $n \ge 4$ and $c \le1/4$, we have $\lg(n!) = \Omega(n\lg n)$.

We can therefore establish $\lg(n!) = \Theta(n\lg n)$ by choosing $c_1 = 1/8$, $c_2 = 2$ and $n_0 = 8$ so that $0 \le c_1 \cdot n\lg n \le \lg(n!) \le c_2 \cdot n\lg n$ for all $n \ge n_0$.

For the second part, let $f(n)=n!$ and $g(n)=n^n$ and consider $\lim_{n \rightarrow \infty} \frac{f(n)}{g(n)}$. If this limit tends to zero, we know that $f(n) = o(g(n))$. For $n \ge 1$, we have:

\[ \frac{n!}{n^n} \le \frac{1}{n} \]

\[ \lim_{n \rightarrow \infty} \frac{n!}{n^n} \le \lim_{n \rightarrow \infty} \frac{1}{n} = 0 \]
Hence we conclude that $n! = o(n^n)$.

\vspace{2mm}
\noindent{
\framebox{
\begin{minipage}{\dimexpr\textwidth-2\fboxsep-2\fboxrule\relax}
\vspace{2mm}
5.  Is the function $\ceil{\lg n}!$ polynomially bounded? Is the function $\ceil{\lg \lg n}!$ polynomially bounded? 
\vspace{2mm}
\end{minipage}
}
}

A function is polynomially bounded if $f(n) = O(n^d)$ for some constant $d$. We must be able to find positive constants $c$ and $n_0$ such that $0 \le f(n) \le c \cdot n^d$ for all $n \ge n_0$. Let $f(n) = \ceil{\lg n}! = \prod_{k=1}^{\ceil{\lg n}} k$, so that:

\[ \prod_{k=\ceil{\lg n}/2}^{\ceil{\lg n}} k \le \prod_{k=1}^{\ceil{\lg n}} k =  f(n) \]

\[ \sum_{k=\ceil{\lg n}/2}^{\ceil{\lg n}} \lg k = \lg\left(\prod_{k=\ceil{\lg n}/2}^{\ceil{\lg n}} k\right) \le   \lg(f(n)) \]

\[ \frac{1}{2}\lg (n)  \cdot \lg \left(\frac{1}{2}\lg n\right) \le \frac{1}{2} \ceil{\lg n} \cdot \lg \left(\frac{1}{2}\ceil{\lg n }\right) \le \lg(f(n)) \]

\[ \lg \left( \sqrt{n}\right) \cdot \lg \lg \left(\sqrt{n} \right) \le \lg (f(n)) \]

\[ \left(2^{\lg \sqrt{n}}\right)^{\lg \lg \sqrt{n}} \le f(n) \]

\[ \sqrt{n}^{\lg \lg \sqrt{n}} \le f(n) \]

If we merely had $\sqrt{n} \le f(n)$, then we could say that $f$ is bounded from below by a polynomial $n^k$ with $k=1/2$. Moreover, the function $\lg \lg \sqrt{n}$ indeed grows very, very slowly, but the fact is that with a lower bound of $\sqrt{n}^{\lg \lg \sqrt{n}}$ there is no polynomial that we can use to bound this function. 

Since $f$ is larger than this function, we know that $f(n)$ grows faster than any polynomial function and therefore is not polynomially bounded.

Now consider the function $f(n) = \ceil{\lg \lg n}!$. Without the loss of generality, we can restrict attention to values of $n$ where $n=2^{2^k}$, where $k \in \mathbb{N}$ since if $n_1=2^{2^{k-1}}$ and $n_2 =2^{2^k}$ then for all $n \in \{n_1+1, n_1+2, ..., n_2\}$, the function $f(n) = \ceil{\lg \lg n}!$ will evaluate to the same value. For the same reason, we can disregard the ceiling function in the analysis that follows. 

To show that $f$ is polynomially bounded, we must find positive constants $c$ and $n_0$ such that $0 \le f(n) \le c \cdot n^d$ for some constant $d$ and for all $n \ge n_0$. Since $n = 2^{2^k}$, we have $\lg \lg n = k$ and hence:

\[ f(n) = \ceil{\lg \lg n}! = k! \le k^k \le c \cdot n^d\]

Taking the logarithm of both sides:

\[ k \lg k \le \lg c + d \lg n \]

Since $\lg n = 2^k$, we have:

\[ k \lg k \le \lg c + d \cdot 2^k\]

The expression $k \lg k \le 2^k$ is true for all $k \ge 1$, so the expression above is true for all $c > 1$ and $d > 1$. For the sake of specificity, we select $c=2$, $d=2$ and $n_0=2$, so that $f(n) = \ceil{\lg \lg n}! \le 2 n^2$ for all $n \ge n_0$, and hence $\ceil{\lg \lg n}!$ is polynomially bounded.

\vspace{2mm}
\noindent{
\framebox{
\begin{minipage}{\dimexpr\textwidth-2\fboxsep-2\fboxrule\relax}
\vspace{2mm}
6.  Which is asymptotically larger: $\lg(\lg^* n)$ or $\lg^*(\lg n)$?
\vspace{2mm}
\end{minipage}
}
}

$\lg^*(\lg n)$ is asymptotically larger than $\lg(\lg^*(n))$. 

\vspace{2mm}
Let $\lg^*(n)= m$, then $\lg(\lg^*(n)) = \lg (m)$. Moreover, the iterated logarithm function simply applies the $\lg(n)$ function repeatedly, $i$ times, while the $\lg^*$ function applies $\lg(n)$ repeatedly until its value is equal to or less than one. Hence, by definition, $\lg^*(\lg(n))=m-1$.

\[ \lg^*(\lg(n)) \rightarrow m-1 \]
\[ \lg(\lg^*(n)) \rightarrow \lg(m) \]
Since $m$ grows asymptotically faster than $\lg(m)$, so we conclude that $\lg^*(\lg(n))$ grows faster than $\lg(\lg^*(n))$.

The following Python code implements the $\log^*$ algorithm:
\begin{lstlisting}[language=Python]
import math

def log2(n):
  return math.log(n)/math.log(2)
  
def logi(n,i):
  if i == 0:
    return n
  return log2(logi(n,i-1))
  
def log_star(n):
  def log_star_iter(k):
    value = logi(n,k)
    if value <= 1.0:
      return k
    return log_star_iter(k+1)
  return log_star_iter(0)
\end{lstlisting}

\vspace{2mm}
\noindent{
\framebox{
\begin{minipage}{\dimexpr\textwidth-2\fboxsep-2\fboxrule\relax}
\vspace{2mm}
7.  Prove by the induction that the $i$th Fibonacci number satisfies the equality $F_i = (\phi^i - \hat{\phi}^i)/\sqrt{5}$, where $\phi$ is the golden ratio and $\hat{\phi}$ is its conjugate. 
\vspace{2mm}
\end{minipage}
}
}

The \textit{golden ratio}  $\phi$ and its conjugate $\hat{\phi}$ are given by:
\[ \phi = \frac{1 + \sqrt{5}}{2}, \hat{\phi} =  \frac{1-\sqrt{5}}{2} \]

The Fibonacci sequence is given by: 

\[ F = \{0, 1, 1, 2, 3, 5, 8, 13, 21, ...\} \]

Suppose we have $i=0$, then $F_0 = 0$, and $\phi^0=1$ and $\hat{\phi}^0 =1$, so $\phi^0 - \hat{\phi}^0=0$ and the condition is satisfied. Suppose next that we have $i=1$, then $F_1 = 1$, and $\phi^1 = (1+\sqrt{5})/2$ and $\hat{\phi}^1 = (1 -\sqrt{5})/2$, so $(\phi^1 - \hat{\phi}^1)/\sqrt{5}$ = $(2\sqrt{5})/(2\sqrt{5})$ = $1$ and the condition is satisfied. 

Suppose next that the condition is satisfied for the $n$th Fibonacci number:

\[ F_n = \frac{\phi^n-\hat{\phi}^n}{\sqrt{5}} \] 

The $n$th Fibonacci number is defined by:

\[ F_n = F_{n-1} + F_{n-2} \]

We proceed by induction, and demonstrate that the condition holds for $n+1$ given that the condition holds for $n$ and $n-1$:

\[ F_{n+1} = F_n + F_{n-1} \]

\[ F_{n+1} = \frac{\phi^n - \hat{\phi}^n}{\sqrt{5}} + \frac{\phi^{n-1} - \hat{\phi}^{n-1}}{\sqrt{5}} \]

\[ F_{n+1} = \frac{\phi^{n-1}\left(\phi + 1\right) - \hat{\phi}^{n-1}(\hat{\phi} + 1)}{\sqrt{5}} \]

We have $\phi + 1 = (3 + \sqrt{5})/2 = (1 + 2\sqrt{5} +5)/4$ = $\{(1 + \sqrt{5})/2\}^2$ = $\phi^2$.

\vspace{2mm}
Likewise $\hat{\phi} + 1 = (3 - \sqrt{5})/2 = (1 - 2\sqrt{5} + 5)/4$ = $\{(1 - \sqrt{5})/2\}^2$ = $\hat{\phi}^2$.

\vspace{2mm}
Hence we can write:

\[ F_{n+1} = \frac{\phi^{n-1} \cdot \phi^2 - \hat{\phi}^{n-1} \cdot \hat{\phi}^2}{\sqrt{5}}  = \frac{\phi^{n+1} - \hat{\phi}^{n+1}}{\sqrt{5}} \]

\vspace{2mm}
\noindent{
\framebox{
\begin{minipage}{\dimexpr\textwidth-2\fboxsep-2\fboxrule\relax}
\vspace{2mm}
8.  Prove that for $i \ge 0$, the $(i+2)$nd Fibonacci number satisfies $F_{i+2} \ge \phi^i$.
\vspace{2mm}
\end{minipage}
}
}

The \textit{golden ratio} is given by:

\[ \phi = \frac{1 + \sqrt{5}}{2} \]

The Fibonacci sequence is given by:

\[ F = \{0, 1, 1, 2, 3, 5, 8, 13, 21, ... \} \]

Suppose that we have $i=0$, then $F_2 = 1$ and $\phi^0 = 1$, and the condition is satisfied. Suppose next that we have $i=1$, then $F_3 = 2$ and $\phi^1 = (1+\sqrt{5})/2$ $\approx$ 1.618, which establishes the result. 

The $(n+2)$the Fibonacci number is given by:

\[ F_{n+2} = F_{n+1} + F_n \]
and by induction we have that $F_{n+1} \ge \phi^{n-1}$ and $F_n \ge \phi^{n-2}$, hence:

\[ F_{n+2} \ge \phi^{n-1} + \phi^{n-2} = \phi^{n-2}\left(\phi + 1\right) \]
In the previous problem we showed that $\phi + 1 = \phi^2$, hence:

\[ F_{n+2} \ge \phi^{n-2} \cdot \phi^2 = \phi^n \]

\end{document}