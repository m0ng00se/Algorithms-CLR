\documentclass[a4paper,12pt]{article}
\usepackage{ amssymb }
\usepackage{mathtools}
\usepackage{amsmath}
\usepackage{amssymb}
\usepackage{listings}
\usepackage{color}
\usepackage[utf8]{inputenc}
\usepackage{amsfonts}
 \usepackage{tikz}
 \usepackage{enumitem}
\DeclarePairedDelimiter\ceil{\lceil}{\rceil}
\DeclarePairedDelimiter\floor{\lfloor}{\rfloor}

\begin{document}

\noindent{
\framebox {
\begin{minipage}{\dimexpr\textwidth-2\fboxsep-2\fboxrule\relax}
\vspace{2mm}
1. Let $A$ and $B$ be finite sets, and let $f : A \rightarrow B$ be a function. Show that

\textbf{a.} if $f$ is injective, then $|A| \le |B|$.

\textbf{b.} if $f$ is surjective, then $|A| \ge |B|$.
\vspace{1mm}
\end{minipage}
}
}

If $f$ is injective, then each element of the domain maps to a unique element in the codomain. 
In other words, if $(a,b) \in f$, then we know there is no other $a' \in A$ where $a \ne a'$ such that $(a',b) \in f$. 
The function $f$ pairs each element of $A$ with exactly one element of $B$.
Hence, we conclude that $|A| \le |B|$.

If $f$ is surjective, then the range of $f$ is equal to its codomain. In other words, for every $b \in B$, we can 
find at least one $a \in A$ such that $(a,b) \in f$. The function $f$ pairs each element of $B$ with at least 
one element in $A$. Hence, we conclude that $|A| \ge |B|$.

\vspace{5mm}
\noindent{
\framebox {
\begin{minipage}{\dimexpr\textwidth-2\fboxsep-2\fboxrule\relax}
\vspace{2mm}
2. Is the function $f(x) = x + 1$ bijective when the domain and the codomain are $\mathbb{N}$? Is it bijective when the domain and the codomain 
are $\mathbb{Z}$?
\vspace{1mm}
\end{minipage}
}
}

$f$ is not bijective when the domain and codomain are $\mathbb{N}$. In this case, $f$ is injective, since $a \ne a' \Rightarrow f(a) \ne f(a')$, but 
$f$ is not surjective. Specifically, as $\mathbb{N} = \{0, 1, 2, ...\}$, there is no $a \in \mathbb{N}$ such that $f(a) = 0$.

$f$ is bijective when the domain and codomain are $\mathbb{Z}$. In this case, $f$ is injective, since $a \ne a' \Rightarrow f(a) \ne f(a')$, and $f$ is also surjective. Specifically $\mathbb{Z} = \{..., -2, -1, 0, 1, 2, ...\}$, so if we are given an arbitrary $b \in \mathbb{Z}$, we can always choose
$a = b -1 \in \mathbb{Z}$ such that $(a,b) = (b-1,b) \in f$.

\vspace{5mm}
\noindent{
\framebox {
\begin{minipage}{\dimexpr\textwidth-2\fboxsep-2\fboxrule\relax}
\vspace{2mm}
3. Give a natural definition for the inverse of a binary relation such that if a relation is in fact a bijective function, its relational inverse is its 
functional inverse.
\vspace{1mm}
\end{minipage}
}
}

[working]

\vspace{5mm}
\noindent{
\framebox {
\begin{minipage}{\dimexpr\textwidth-2\fboxsep-2\fboxrule\relax}
\vspace{2mm}
4. Give a bijection from $\mathbb{Z}$ to $\mathbb{Z} \times \mathbb{Z}$.
\vspace{1mm}
\end{minipage}
}
}

[working]

\end{document}