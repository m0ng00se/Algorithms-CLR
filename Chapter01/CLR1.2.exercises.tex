\documentclass[a4paper,12pt]{article}
\usepackage{ amssymb }
\usepackage{mathtools}
\usepackage{listings}
\usepackage{color}
\usepackage[utf8]{inputenc}
\DeclarePairedDelimiter\ceil{\lceil}{\rceil}
\DeclarePairedDelimiter\floor{\lfloor}{\rfloor}

\definecolor{codegreen}{rgb}{0,0.6,0}
\definecolor{codegray}{rgb}{0.5,0.5,0.5}
\definecolor{codepurple}{rgb}{0.58,0,0.82}
\definecolor{backcolour}{rgb}{0.95,0.95,0.92}
\definecolor{highlightcolor}{rgb}{0.8, 0.9, 0.9}

\lstdefinestyle{mystyle}{
    backgroundcolor=\color{backcolour},   
    commentstyle=\color{codegreen},
    keywordstyle=\color{magenta},
    numberstyle=\tiny\color{codegray},
    stringstyle=\color{codepurple},
    basicstyle=\footnotesize,
    breakatwhitespace=false,         
    breaklines=true,                 
    captionpos=b,                    
    keepspaces=true,                 
    numbers=left,                    
    numbersep=5pt,                  
    showspaces=false,                
    showstringspaces=false,
    showtabs=false,                  
    tabsize=2
}
 
\lstset{style=mystyle}

\newcommand*\Suppressnumber{%
  \lst@AddToHook{OnNewLine}{%
    \let\thelstnumber\relax%
     \advance\c@lstnumber-\@ne\relax%
    }%
}

\begin{document}

\noindent{
\framebox {
\begin{minipage}{\dimexpr\textwidth-2\fboxsep-2\fboxrule\relax}
\vspace{2mm}
4. Consider the problem of evaluating a polynomial at a point. Given $n$ coefficients $a_0, a_1, ..., a_n$ and a real number $x$, we wish to compute $\sum_{ii=o}^{n-1}a_ix_i$. Describe a straightforward $\Theta(n^2)$-time algorithm for this problem. Describe a $\Theta(n)$-time algorithm that uses the following method (called Horner's rule) for rewriting the polynomial:

\[ \sum_{i=0}^{n-1}a_ix^i = (...(a_{n-1}x + a_{n_2})x + ... + a_1)x + a_0 \]
\vspace{2mm}
\end{minipage}
}
}

Computation of polynomials requires multiplication and addition. Let us assume, for our model of computation, that each multiplication and addition costs a constant amount of time, $c_m$ and $c_a$ respectively. 

The computation of $x^i$ is bounded from above by $\Theta(n)$  (i.e., a maximum of $n$ multiplications). There are a total $n$ such terms in the expression. Hence we should
be able to evaluate the expression in $\Theta(n^2)$ time, in the worst case. 

An example algorithm is as follows:

\begin{lstlisting}[language=Java]
import java.util.List;

\end{lstlisting}

\vspace{5mm}

\noindent{
\framebox {
\begin{minipage}{\dimexpr\textwidth-2\fboxsep-2\fboxrule\relax}
\vspace{2mm}
5. Express the function $n^3/1000 - 100n^2 - 100n + 3$ in terms of $\Theta$-notation.
\vspace{2mm}
\end{minipage}
}
}

For a polynomial function $f$ in $n$, only the highest order term is relevant when considering order of growth statistics. More precisely, suppose that 

\[ f(n) = \sum_{i=0}^k a_in^i = a_kn^k + a_{k-1}n^{k-1} + ... + a_0 \]

\[ \frac{1}{n^k} f(n) = a_k + \frac{1}{n} a_{k-1} + ...  + \frac{1}{n^k}a_0  \]

so that

\[ \lim_{n \rightarrow \infty} \frac{1}{n^k}f(n) = a_k \]

Hence for large $n$ we can write

\[ f(n) \approx a_k n^k \]

The constant coefficient $a_k$ can be ignored, giving $\Theta(f(n)) = n^k$.

In the example above, where $f(n) = n^3/1000 - 100n^2 - 100n + 3$, even though the quadratic and linear
terms have large negative coefficients and even though the coefficient on the $n^3$ term is a small fraction,
we can still write $\Theta(f(n)) = n^3$.

\vspace{5mm}

\noindent{
\framebox {
\begin{minipage}{\dimexpr\textwidth-2\fboxsep-2\fboxrule\relax}
\vspace{2mm}
6. How can we modify almost any algorithm to have a good best-case running time?
\vspace{2mm}
\end{minipage}
}
}

Working..

\end{document}