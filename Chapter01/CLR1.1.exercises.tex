\documentclass[a4paper,12pt]{article}
\usepackage{ amssymb }
\usepackage{mathtools}
\usepackage{listings}
\usepackage{color}
\usepackage[utf8]{inputenc}
\DeclarePairedDelimiter\ceil{\lceil}{\rceil}
\DeclarePairedDelimiter\floor{\lfloor}{\rfloor}

\definecolor{codegreen}{rgb}{0,0.6,0}
\definecolor{codegray}{rgb}{0.5,0.5,0.5}
\definecolor{codepurple}{rgb}{0.58,0,0.82}
\definecolor{backcolour}{rgb}{0.95,0.95,0.92}
\definecolor{highlightcolor}{rgb}{0.8, 0.9, 0.9}

\lstdefinestyle{mystyle}{
    backgroundcolor=\color{backcolour},   
    commentstyle=\color{codegreen},
    keywordstyle=\color{magenta},
    numberstyle=\tiny\color{codegray},
    stringstyle=\color{codepurple},
    basicstyle=\footnotesize,
    breakatwhitespace=false,         
    breaklines=true,                 
    captionpos=b,                    
    keepspaces=true,                 
    numbers=left,                    
    numbersep=5pt,                  
    showspaces=false,                
    showstringspaces=false,
    showtabs=false,                  
    tabsize=2
}
 
\lstset{style=mystyle}

\newcommand*\Suppressnumber{%
  \lst@AddToHook{OnNewLine}{%
    \let\thelstnumber\relax%
     \advance\c@lstnumber-\@ne\relax%
    }%
}

\begin{document}

\noindent{
\framebox {
\begin{minipage}{\dimexpr\textwidth-2\fboxsep-2\fboxrule\relax}
\vspace{2mm}
1.  Using Figure 1.2 as a model, illustrate the operation of INSERTION-SORT on the array $A = <31, 41, 59, 26, 41, 58>$.
\vspace{2mm}
\end{minipage}
}
}

The following Java code implements INSERTION-SORT:

\begin{lstlisting}[language=Java]
import java.util.List;

public class Sort {
  public <T extends Comparable<T>> void sort(List<T> m) {
    //! Guard against IndexOutOfBoundsException
    if ( m.size() <= 1 )
      return;
      
    //! Implement INSERTION-SORT
    for ( int i=1; i < m.size(); ++i ) {
      T elem = m.get(i);
      int j = i-1;
      while ( j >= 0 && m.get(j).compareTo(elem) > 0 ) {
        m.set(j+1, m.get(j)); j--;
      }
      m.set(j+1, elem);
    }
  }
}
\end{lstlisting}

This is an in-place implementation of INSERTION-SORT, with the output overwriting the input and at most a constant amount of secondary memory being allocated from the heap. 
The following illustrates the state of the input array as the algorithm runs:

\vspace{5mm}
$i=1$, $j=0$, $elem = 41$, $A = <31, 41, 59, 26, 41, 58>$

$i=1$, $j=0$, $elem = 41$, $A = <31, \colorbox{highlightcolor}{41}, 59, 26, 41, 58>$ $\rightarrow$ $A(1) = 41$

$i=2$, $j=1$, $elem = 59$, $A = <31, 41, 59, 26, 41, 58>$

$i=2$, $j=1$, $elem = 59$, $A = <31, 41, \colorbox{highlightcolor}{59}, 26, 41, 58>$ $\rightarrow$ $A(2) = 59$

$i=3$, $j=2$, $elem = 26$, $A = <31, 41, 59, 26, 41, 58>$ 

$i=3$, $j=2$, $elem = 26$, $A = <31, 41, 59, \colorbox{highlightcolor}{59}, 41, 58>$ $\rightarrow$ $A(3) = 59$

$i=3$, $j=1$, $elem = 26$, $A = <31, 41, \colorbox{highlightcolor}{41}, \colorbox{highlightcolor}{59}, 41, 58>$ $\rightarrow$ $A(2) = 41$
 
$i=3$, $j=0$, $elem = 26$, $A = <31,\colorbox{highlightcolor}{31}, \colorbox{highlightcolor}{41}, \colorbox{highlightcolor}{59}, 41, 58>$ $\rightarrow$  $A(1) = 31$

$i=3$, $j=-1$, $elem = 26$, $A = <\colorbox{highlightcolor}{26}, \colorbox{highlightcolor}{31}, \colorbox{highlightcolor}{41}, \colorbox{highlightcolor}{59}, 41, 58>$  $\rightarrow$ $A(0) = 26$

$i=4$, $j=3$, $elem = 41$, $A = <26, 31, 41, 59, 41, 58>$

$i=4$, $j=3$, $elem = 41$, $A = <26, 31, 41, 59, \colorbox{highlightcolor}{59}, 58>$ $\rightarrow$ $A(4) = 59$

$i=4$, $j=2$, $elem = 41$, $A = <26, 31, 41, \colorbox{highlightcolor}{41}, \colorbox{highlightcolor}{59}, 58>$ $\rightarrow$ $A(3) = 41$

$i=5$, $j=4$, $elem = 58$, $A = <26, 31, 41, 41, 59, 58>$

$i=5$, $j=4$, $elem = 58$, $A = <26, 31, 41, 41, 59, \colorbox{highlightcolor}{59}>$ $\rightarrow$ $A(5) = 59$

$i=5$, $j=4$, $elem = 58$, $A = <26, 31, 41, 41, \colorbox{highlightcolor}{58}, \colorbox{highlightcolor}{59}>$ $\rightarrow$ $A(4) = 58$

\newpage
The final result is therefore:

\[ A = <26, 31, 41, 41, 58, 59> \]

\vspace{5mm}
\noindent{
\framebox {
\begin{minipage}{\dimexpr\textwidth-2\fboxsep-2\fboxrule\relax}
\vspace{2mm}
2. Rewrite the INSERTION-SORT procedure to sort into non-increasing instead of non-decreasing order. 
\vspace{2mm}
\end{minipage}
}
}

The following Java code provides an implementation of INSERTION-SORT that sorts arrays into non-increasing order:

\begin{lstlisting}[language=Java]
import java.util.List;

public class Sort {
  public <T extends Comparable<T>> void sort(List<T> m) {
    //! Guard against IndexOutOfBoundsException
    if ( m.size() <= 1 ) 
      return m;
      
    //! Implement INSERTION-SORT
    for ( int i=1; i < m.size(); ++i ) {
      T elem = m.get(i);
      int j = i-1;
      while ( j >= 0 && m.get(j).compareTo(elem) < 0 ) {
        m.set(j+1, m.get(j)); j--;
      }
      m.set(j+1, elem);
    }
  }
}
\end{lstlisting}

The only change is at line 11 where

\begin{lstlisting}[numbers=none]
m.get(j).compareTo(elem) > 0
\end{lstlisting}

\noindent
has been changed to:

\begin{lstlisting}[numbers=none]
m.get(j).compareTo(elem) < 0 
\end{lstlisting}

\vspace{5mm}

\noindent{
\framebox {
\begin{minipage}{\dimexpr\textwidth-2\fboxsep-2\fboxrule\relax}
\vspace{2mm}
3. Consider the \textit{\textbf{searching problem}}:

\vspace{2mm}
\textbf{Input}:  A sequence of $n$ numbers $A = <a_1, a_2, ..., a_n>$ and a value $v$. 

\vspace{2mm}
\textbf{Output}: An index $i$ such that $v = A[i]$  or the special value NIL if $v$ does not appear in $A$.

\vspace{2mm}
Write pseudocode for \textit{\textbf{linear search}}, which scans through the sequence looking for $v$.
\vspace{2mm}
\end{minipage}
}
}

\newpage
The following Python code implements the algorithm:

\vspace{3mm}
\begin{lstlisting}[language=Python]
def search(A, v):
  j = None
  for i in range(len(A)):
    if A[i] == v:
      j = i
      break;   
  return j
\end{lstlisting}

\vspace{5mm}
\noindent{
\framebox {
\begin{minipage}{\dimexpr\textwidth-2\fboxsep-2\fboxrule\relax}
\vspace{2mm}
4. Consider the problem of adding two $n$-bit binary numbers, stored in two $n$-element arrays
$A$ and $B$. The sum of the two integers should be stored in an $(n+1)$-element array $C$. State the problem formally and write pseudocode for adding the two integers.
\vspace{2mm}
\end{minipage}
}
}

We can reason about this problem inductively (i.e., recursively). 

\vspace{2mm}
Suppose we start by adding two $1$-bit binary numbers. 

\vspace{2mm}
There are four cases to consider:

\vspace{2mm}
[0] + [0] = [00] \newline  \indent
[1] + [0] = [01] \newline \indent
[0] + [1] = [01] \newline \indent
[1] + [1] = [10] 

\vspace{2mm}
From this pattern we can surmise that:

\[ C[i] = (A[i] + B[i]) \% 2 \] 
\[ C[i+1] = (A[i] + B[i]) / 2 \]

We can implement the desired algorithm in Python as follows:

\vspace{3mm}
\begin{lstlisting}[language=Python]
def add(A, B, n):
  A_ = A[:]; A_.reverse()
  B_ = B[:]; B_.reverse()
  C = [0] * (n+1)
  for i in range(n):
    sum_ = A_[i] + B_[i] + C[i]
    C[i] = int(sum_%2)
    C[i+1] = int(sum_/2)
  C.reverse()
  return C
\end{lstlisting}

\end{document}