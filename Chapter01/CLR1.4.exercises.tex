\documentclass[a4paper,12pt]{article}
\usepackage{ amssymb }
\usepackage{mathtools}
\usepackage{listings}
\usepackage{color}
\usepackage[utf8]{inputenc}
\usepackage{amsmath}
\DeclarePairedDelimiter\ceil{\lceil}{\rceil}
\DeclarePairedDelimiter\floor{\lfloor}{\rfloor}

\definecolor{codegreen}{rgb}{0,0.6,0}
\definecolor{codegray}{rgb}{0.5,0.5,0.5}
\definecolor{codepurple}{rgb}{0.58,0,0.82}
\definecolor{backcolour}{rgb}{0.95,0.95,0.92}
\definecolor{highlightcolor}{rgb}{0.8, 0.9, 0.9}

\lstdefinestyle{mystyle}{
    backgroundcolor=\color{backcolour},   
    commentstyle=\color{codegreen},
    keywordstyle=\color{magenta},
    numberstyle=\tiny\color{codegray},
    stringstyle=\color{codepurple},
    basicstyle=\footnotesize,
    breakatwhitespace=false,         
    breaklines=true,                 
    captionpos=b,                    
    keepspaces=true,                 
    numbers=left,                    
    numbersep=5pt,                  
    showspaces=false,                
    showstringspaces=false,
    showtabs=false,                  
    tabsize=2
}
 
\lstset{style=mystyle}

\newcommand*\Suppressnumber{%
  \lst@AddToHook{OnNewLine}{%
    \let\thelstnumber\relax%
     \advance\c@lstnumber-\@ne\relax%
    }%
}

\begin{document}

\noindent{
\framebox {
\begin{minipage}{\dimexpr\textwidth-2\fboxsep-2\fboxrule\relax}
\vspace{2mm}
1. Suppose we are comparing implementations of insertion sort and merge sort on 
the same machine. For inputs of size $n$, insertion sort runs in $8n^2$ steps, while 
merge sort runs in $64n \lg n$ steps. For which values of $n$ does insertion sort beat
merge sort? How might one rewrite the merge sort procedure to make it 
even faster on small inputs?
\vspace{2mm}
\end{minipage}
}
}

We wish to find $n_0$ such that for $n > n_0$, the following inequality holds:
\[ 8n^2 < 64n \lg n\]

Defining $f(n) = 64n \lg n - 8n^2$, this is equivalent to finding values of $n$ for which 
$f(n) > 0$. Newton's method is a useful numerical technique for calculating the roots 
of transcendental functions like this one. Starting with an initial guess $x_0$, the 
subsequent (hopefully better) guess is given by:

\[ x_1 = x_0 - \frac{f(x_0)}{f'(x_0)} \]

In this case, $f(n) = 64n \lg n - 8n^2$ and $f'(n) = 64\lg n + 64- 16n$. The following Python script calculates a numerical 
approximation for the roots of this function using Newton's method:

\begin{lstlisting}[language=Python]
import math

def newton(F, dF, x):
    """ Return next iterative approximation. """
    return x - F(x)/dF(x)

def F(x):
    """ Value of 64x*lg(x) - 8x^2 """
    return 64*x*math.log(x) - 8*math.pow(x,2)
    
def dF(x):
    """ Value of derivative of 64x*lg(x) - 8x^2. """
    return 64*math.log(x) + 64 - 16*x

def find_root(F, dF, initial_guess, tolerance):
    """ Find root using Newton's method to specified tolerance, using initial guess. """
    def guess(x0):
        x1 = newton(F, dF, x0)
        diff = abs(x0-x1)
        return (x1,diff)

    # Iterate until we obtain a result within tolerance.                                                   
    (x1,diff) = guess(initial_guess)
    while (diff > tolerance):
        (x1, diff) = guess(x1)
    return x1

# Find the root                                                                                            
print find_root(F, dF, 20, 0.01)
\end{lstlisting}

The script returns 26.0934 as an approximate root, which indeed is valid:
\[ 8 * 26 * 26 = 5,408 < 5,421.47 \approx 64 * 26 * \lg(26) \]
\[ 8 * 27 * 27 = 5,832 > 5,695.21 \approx 64 * 27 * \lg(27) \]

The answer to the question is that for inputs $n < 27$, insertion sort will 
run faster than merge sort.

[Working]

\vspace{5mm}

\noindent{
\framebox {
\begin{minipage}{\dimexpr\textwidth-2\fboxsep-2\fboxrule\relax}
\vspace{2mm}
2. What is the smallest value of $n$ such that an algorithm whose running time is $100n^2$ runs faster than an algorithm
whose running time is $2^n$ on the same machine?
\vspace{2mm}
\end{minipage}
}
}

We wish to find $n_0$ such that for $n > n_0$ the following inequality holds:
\[ 100n^2 < 2^n \]

Defining $f(n) = 2^n - 100n^2$, this is equivalent to finding values of $n$ for which $f(n) > 0$.
Newton's method is a useful numerical technique for calculating the roots of transcendental functions like this one.
Starting with an initial guess $x_0$, the subsequent (hopefully better) guess is given by:
\[ x_1 = x_0 - \frac{f(x_0)}{f'(x_0)} \]

In this case, $f(n) = 2^n - 100n^2$ and $f'(n) = (\ln 2) \cdot 2^n - 200 n$.
The following Python script calculates a numerical approximation for the roots of this function using Newton's method:


\begin{lstlisting}[language=Python]
import math

def newton(F, dF, x):
    """ Return next iterative approximation. """
    return x - F(x)/dF(x)
    
def F(x):
    """ Value of 2^x - 100x^2. """
    return math.pow(2,x) - 100*math.pow(x,2)
    
def dF(x):
    """ Value of derivative of 2^x - 100x^2. """
    return math.log(2)*math.pow(2,x) - 200*x
    
def find_root(F, dF, initial_guess, tolerance):
    """ Find root using Newton's method. """
    def guess(x0):
        x1 = newton(F, dF, x0)
        diff = abs(x0-x1)
        return (x1,diff)

    # Iterate until we obtain a result within tolerance                                                   
    (x1,diff) = guess(initial_guess)
    while (diff > tolerance):
        (x1, diff) = guess(x1)
    return x1
    
# Find the root with initial guess of 20                                                                                            
print find_root(F, dF, 20, 0.01)
\end{lstlisting}

The script returns $14.3247$ as an approximate root, which indeed is valid:
\[ 100 * 14 * 14 = 19,600 > 16,384 = 2^{14} \]
\[ 100 * 15 * 15 = 22,500 < 32,768 = 2^{15} \]

The answer to the question is that for $n=15$ and higher the $T(n) = 100n^2$ procedure will 
outperform the $T(n) = 2^n$ procedure. 

The problem illustrates that an exponential procedure 
grows much more rapidly than a quadratic function, 
and that even for $n = 15$ a very bad quadratic procedure will greatly outperform an exponential one.

\end{document}