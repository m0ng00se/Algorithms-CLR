\documentclass[a4paper,12pt]{article}
\usepackage{ amssymb }
\usepackage{mathtools}
\usepackage{amsmath}
\usepackage{amssymb}
\usepackage{listings}
\usepackage{color}
\usepackage[utf8]{inputenc}
\usepackage{amsfonts}
 \usepackage{tikz}
\DeclarePairedDelimiter\ceil{\lceil}{\rceil}
\DeclarePairedDelimiter\floor{\lfloor}{\rfloor}

\begin{document}

\noindent{
\framebox{
\begin{minipage}{\dimexpr\textwidth-2\fboxsep-2\fboxrule\relax}
\vspace{2mm}
1. Find a simple formula for $\sum_{k=1}^n (2k-1)$.
\vspace{2mm}
\end{minipage}
}
}

By distribution: 

\[ \sum_{k=1}^n (2k-1) = 2\sum_{k=1}^n k - \sum_{k=1}^n  1\]

Since $\sum_{k=1}^n k = n(n+1)/2$ and $\sum_{k=1}^n a = n a$, we can write:

\[ = 2\cdot\left( \frac{1}{2} \right) \cdot n \left(n+1\right) - n\]

\[ = n^2 + n - n = n^2 \]

Hence:

\[ S(n) = \sum_{k=1}^n (2k-1)  = n^2 \]

\noindent{
\framebox{
\begin{minipage}{\dimexpr\textwidth-2\fboxsep-2\fboxrule\relax}
\vspace{2mm}
2. Show that $\sum_{k=1}^n 1 / \left(2k-1\right) = \ln(\sqrt{n}) + O(1)$ by manipulating the harmonic series.
\vspace{2mm}
\end{minipage}
}
}

We can define $K(n)$ as:

\[ K(n) = \sum_{k=1}^n 1 / \left(2k-1\right) = 1 + \frac{1}{3} + \frac{1}{5} + ... + \frac{1}{2n-1}\]

so that the harmonic function

\[ H(n) = 1 + \frac{1}{2} + \frac{1}{3} + ... + \frac{1}{n} = \ln(n) + O(1) \]

can be expressed as: 

\[ H(2n) = \frac{1}{2} H(n) + K(n) \]

\[ K(n) = H(2n) - \frac{1}{2} H(n) \]

Substituting the expression from above for $H(n)$ and $H(2n)$, we have:

\[ K(n) =  \ln(2n) + O(1) - \frac{1}{2} \left\{ \ln(n) + O(1) \right\}  \]

\[ K(n) = \ln(2n) - \frac{1}{2}\ln(n) + \frac{1}{2} \cdot O(1) \] 

Since $\ln(2n) = \ln(2) + \ln(n)$, we can write: 

\[ K(n) = \frac{1}{2} \ln(n) + \frac{1}{2}\cdot O(1) + \ln(2) \]

With asymptotic notation, the expression $\frac{1}{2} \cdot O(1)$ reduces to $O(1)$, and the constant $\ln(2)$ can be absorbed into the $O(1)$ term as well:

\[ K(n) = \frac{1}{2}\log(n) + O(1)  \]

\[ K(n) = \log(\sqrt{n}) + O(1) \]

\[ \sum_{k=1}^n \frac{1}{2k-1} = \log(\sqrt{n}) + O(1) \]

\noindent{
\framebox{
\begin{minipage}{\dimexpr\textwidth-2\fboxsep-2\fboxrule\relax}
\vspace{2mm}
3. Show that $\sum_{k=0}^\infty(k-1)/2^k = 0$.
\vspace{2mm}
\end{minipage}
}
}

Since $|\frac{k-1}{2^k}| < 1$ for $k > 1$, we can write:

\[ S(x) = \sum_{k=0}^\infty x^k = \frac{1}{1-x} \]

\[ S'(x) = \sum_{k=0}^\infty k x^{k-1} = \frac{1}{(1-x)^2} \]

\[ xS'(x) = \sum_{k=0}^\infty k x^{k} = \frac{x}{(1-x)^2}  \]

Next we define $F(x)$ as:

\[ F(x) = xS'(x) - S(x) = \frac{x}{(1-x)^2} - \frac{1}{1-x} \]

\[ F(x) = \sum_{k=0}^\infty k x^k - \sum_{k=0}^\infty x^k = \frac{2x-1}{(1-x)^2}\]

\[ F(x) = \sum_{k=0}^\infty (k-1)x^k = \frac{2x-1}{(1-x)^2} \]

We can now write:

\[ F(1/2) = \sum_{k=0}^\infty (k-1)/2^k = \frac{2(1/2) - 1}{1/4} = 0 \]

\[ \sum_{k=0}^\infty (k-1)/2^k  = 0 \]

\noindent{
\framebox{
\begin{minipage}{\dimexpr\textwidth-2\fboxsep-2\fboxrule\relax}
\vspace{2mm}
4. Evaluate the sum $\sum_{k=1}^\infty (2k+1) x^{2k}$.
\vspace{2mm}
\end{minipage}
}
}

We can write:

\[ \sum_{k=1}^\infty x^k = \frac{x}{1-x} \]

Define $S(x)$ as:
\[ S(x) = \sum_{k=1}^\infty (x^2)^k = \sum_{k=1}^\infty x^{2k} = \frac{x^2}{1-x^2} \]

so that

\[ S'(x) = \sum_{k=1}^\infty (2k)x^{2k-1} = \frac{2x}{(1-x^2)^2} \]

\[ xS'(x) = \sum_{k=1}^\infty (2k)x^{2k} = \frac{2x^2}{(1-x^2)^2}  \]

Now define $F(x)$ as:

\[ F(x) = xS'(x) + S(x) = \sum_{k=1}^\infty (2k)x^{2k} + \sum_{k=1}^\infty x^{2k} = \frac{2x^2}{(1-x^2)^2} + \frac{x^2}{1-x^2} \]

\[ F(x) = \sum_{k=1}^\infty(2k+1) x^{2k} = \frac{3x^2 - x^4}{(1-x^2)^2} \] 

\noindent{
\framebox{
\begin{minipage}{\dimexpr\textwidth-2\fboxsep-2\fboxrule\relax}
\vspace{2mm}
5. Use the linearity property of summations to prove that:
\[ \sum_{k=1}^n O(f_k(m)) = O\left(\sum_{k=1}^n f_k(m)\right) \]
\vspace{0mm}
\end{minipage}
}
}

By definition, a function $h(m)$ is $O(f(m))$, written  $h(m) = O(f(m))$, if we can find positive constants $c$ and $m_0$ such that $0 \le h(m) \le c \cdot f(m)$ of all $m \ge m_0$.

We can therefore expand $\sum_{k=1}^n O(f_k(m))$ as follows:

\[ \sum_{k=1}^n O\left(f_k(m)\right) = O(f_1(m)) + O(f_2(m)) + ... + O(f_n(m)) \]

\[ = h_1(m) + h_2(m) + ... + h_n(m) \]
where

\[ h_k(m) = O(f_k(m)) \]

For each $h_k(m)$ we can, by definition, find positive constants $c_k$ and $m_k$ such that $h_k(m) \le c_k \cdot f_k(m)$ for all $m \ge m_k$.

Let $m_0 = \max\left(m_k\right)$, and define $h(m)$ as:

\[ h(m) = \sum_{k=1}^n O \left(f_k(m)\right) \]

Provided that $m \ge m_0$, we can now write:

\[ h(m) \le \sum_{k=1}^n c_k \cdot f_k(m) \]

Let $c_0 = \max(c_k)$, so that we can write:

\[ h(m) \le \sum_{k=1}^n c_k \cdot f_k(m) \le \sum_{k=1}^n c_0 \cdot f_k(m) \]

By the linearity property:

\[ h(m) \le c_0 \cdot  \sum_{k=1}^n f_k(m) \]

By construction, we have identified positive constants $c_0$ and $m_0$ such that the expression above holds true. 

Consequently, we can write:

\[ h(m) = O\left(\sum_{k=1}^n f_k(m) \right)\]

which was to be proved. 

\vspace{5mm}
\noindent{
\framebox{
\begin{minipage}{\dimexpr\textwidth-2\fboxsep-2\fboxrule\relax}
\vspace{2mm}
6. Prove that:

\[ \sum_{k=1}^\infty \Omega (f_k(m)) = \Omega \left(\sum_{k=1}^\infty f_k(m)\right) \]
\vspace{0mm}
\end{minipage}
}
}

We presume that the limit exists, otherwise the problem is ill-posed. 

By definition, a function $h(m)$ can be expressed as $h(m) = \Omega(f(m))$ if we can find positive constants $c$ and $m_0$ such that $0 \le c\cdot f(m) \le h(m)$ for all $m \ge m_0$.

We can therefore expand $\sum_{k=1}^n \Omega(f_k(m))$ as follows:

\[ \sum_{k=1}^n \Omega(f_k(m)) = \Omega(f_1(m)) + \Omega(f_2(m)) + ... + \Omega(f_n(m)) \]

\[ = h_1(m) + h_2(m) + ... + h_n(m) \]

For each $h_k(m)$ we can, by definition, find positive constants $c_k$ and $m_k$ such that $c_k \cdot f_k(m) \le h_k(m)$ for all $m \ge m_k$.

Let $m_0 = \max(m_k)$, and define $h(m)$ as:

\[ h(m) = \lim_{n \rightarrow \infty} \sum_{k=1}^n \Omega\left(f_k(m)\right) = \sum_{k=1}^\infty \Omega\left( f_k(m)\right) \]

Provided that $m \ge m_0$, we can now write:

\[ \lim_{n \rightarrow \infty} \sum_{k=1}^n c_k \cdot f_k(m) \le h(m) \]

Let $c_0 = \min(c_k)$, so that we can write:

\[  \lim_{n \rightarrow \infty} \sum_{k=1}^n c_0 \cdot f_k(m) \le \lim_{n \rightarrow \infty} \sum_{k=1}^n c_k \cdot f_k(m) \le h(m) \]

By the linearity property: 

\[ c_0 \cdot \lim_{n \rightarrow \infty} \sum_{k=1}^n f_k(m) =  c_0 \cdot \sum_{k=1}^\infty f_k(m) \le h(m) =  \sum_{k=1}^\infty \Omega (f_k(m))  \]

By construction, we have identified positive constants $c_0$ and $m_0$ such that the expression above holds true. 

Consequently, we can write:

\[ \sum_{k=1}^\infty \Omega (f_k(m))  = \Omega \left(\sum_{k=1}^\infty f_k(m) \right) \]

which was to be proven.

\vspace{5mm}
\noindent{
\framebox{
\begin{minipage}{\dimexpr\textwidth-2\fboxsep-2\fboxrule\relax}
\vspace{2mm}
7. Evaluate the product $\prod_{k=1}^n 2 \cdot 4^k$.
\vspace{2mm}
\end{minipage}
}
}

We can factor out the 2, which will contribute $2^n$ to the final product.

\vspace{2mm}
Define $P(n)$ as:

\[ P(n) = \prod_{k=1}^n 4^k = \prod_{k=1}^n 2^{2k} \]

\[ \lg P(n) = \sum_{k=1}^n \lg 2^{2k} = \sum_{k=1}^n 2k \]

\[ \lg P(n) = n(n+1) \]

\[ P(n) = 2^{n(n+1)} \]

Factoring back in $2^n$ term, we are left with:

\[ \prod_{k=1}^n 2 \cdot 4^k = 2^{n(n+2)} \]

\noindent{
\framebox{
\begin{minipage}{\dimexpr\textwidth-2\fboxsep-2\fboxrule\relax}
\vspace{2mm}
8. Evaluate the product $\prod_{k=2}^n \left(1 - 1/k^2\right)$.
\vspace{2mm}
\end{minipage}
}
}

Define $P(n)$ as:

\[ P(n) = \prod_{k=2}^n \left(1 - 1/k^2\right) = \prod_{k=2}^n\left( \frac{k^2-1}{k^2} \right) = \prod_{k=2}^n \left\{ \frac{(k+1)(k-1)}{k^2}\right\} \]

\[ \lg P(n) = \sum_{k=2}^n \left\{\lg (k+1) + \lg(k-1) - 2\lg k \right\}\]

\[ = \sum_{k=2}^n \left\{ \lg(k+1) - \lg k \right\}+ \sum_{k=2}^n \left\{\lg(k-1) - \lg k \right\} \]

We evaluate each of these two terms separately:

\[ S_1(n) = \sum_{k=2}^n \left\{\lg(k+1) - \lg k\right\} \]

\[ = -\sum_{k=2}^n \left\{\lg k - \lg(k+1) \right\} \]

\[ = -\left\{\ln 2 - \ln(n+1)\right\}\]

\[ = \ln\left(\frac{n+1}{2}\right) \]

and

\[ S_2(n) = \sum_{k=2}^n \left\{\lg(k-1) - \lg k \right\} \]

\[ = - \sum_{k=2}^n \left\{\lg k - \lg(k-1) \right\} \]

\[ = -\left\{\ln n - \ln 1\right\}\]

\[ = -\ln n \]

so that

\[ \lg P(n) = S_1(n) + S_2(n) = \ln \left(\frac{n+1}{2}\right) - \ln n \]

\[ \lg P(n) = \ln \left(\frac{n+1}{2n} \right) \]

\[ P(n) = \frac{n+1}{2n} \]

so that

\[ \prod_{k=2}^n \left(1 - 1/k^2\right) = \frac{n+1}{2n}\]

\end{document}